\documentclass[letterpaper,12pt]{article}
\usepackage[utf8]{inputenc}
\usepackage{amsmath, amsfonts}
\usepackage{amsthm}
\usepackage{braket}

% remove spacing around date and author
\usepackage{titling}
\predate{}
\postdate{}
\preauthor{}
\postauthor{}

\author{}
\title{MATH 262 - Homework 4.3}
\date{} % clear date

\theoremstyle{remark}
\newtheorem*{claim}{Claim}

\begin{document}

\maketitle

\begin{enumerate}
  \item[6.]
    \begin{claim}
      Let $T: V \to W$ be a linear transformation of linear spaces $V$ and $W$, suppose that a set of vectors $\Set{\vec{v_1}, \vec{v_2}, \vec{v_3}} \subset V$ has the property that $\Set{T(\vec{v_1}), T(\vec{v_2}), T(\vec{v_3})} \subset W$ is linearly independent. Then, the set of vectors $\Set{\vec{v_1}, \vec{v_2}, \vec{v_3}}$ is linearly independent.
    \end{claim}
    \begin{proof}
      Suppose that $a_1\vec{v_1} + a_2\vec{v_2} + a_3\vec{v_3} = \vec{0}$. \\
      Then, applying $T$ to both sides we get
      \begin{equation*}
        T(a_1\vec{v_1} + a_2\vec{v_2} + a_3\vec{v_3}) = T(\vec{0})
      \end{equation*}
      Since $T$ is a linear transformation, we get
      \begin{equation*}
        a_1T(\vec{v_1}) + a_2T(\vec{v_2}) + a_3T(\vec{v_3}) = T(\vec{0})
      \end{equation*}
      Since $\Set{T(\vec{v_1}), T(\vec{v_2}), T(\vec{v_3})}$ are linearly independent we get that
      \begin{equation*}
        a_1T(\vec{v_1}) + a_2T(\vec{v_2}) + a_3T(\vec{v_3}) = \vec{0}
      \end{equation*}
      This also means the only solution is the trivial $\vec{a} = \vec{0}$. Notice the supposition made at the beginning also uses $\vec{a}$. Thus, that supposition also only has $\vec{a} = \vec{0}$ as a solution and $\vec{v_1}$, $\vec{v_2}$, and $\vec{v_3}$ are also linearly independent.
    \end{proof}
\end{enumerate}

\end{document}
