\documentclass[letterpaper,12pt]{article}
\usepackage[utf8]{inputenc}
\usepackage{amsmath, amsfonts}
\usepackage[x11names]{xcolor}
\usepackage{pgfplots}
\usepackage{tikz}
\usepackage{braket}

\usetikzlibrary{arrows.meta}
\pgfplotsset{
  compat = newest,
  blank_style/.style={
    clip=false,
    axis equal,
    axis line style={draw=none},
    tick style={draw=none},
    ticks=none,
    line width=1pt,
  },
  blank_clamped_style/.style={
    blank_style,
    xmin=-0.5,
    xmax=1,
    ymin=-1,
    ymax=1,
  },
}

% remove spacing around date and author
\usepackage{titling}
\predate{}
\postdate{}
\preauthor{}
\postauthor{}

\author{}
\title{MATH 262 - Homework 7.1}
\date{} % clear date

\begin{document}

\maketitle

\begin{enumerate}
  \item[4.]
    Arguing geometrically, find all eigenvectors and eigenvalues of the linear transformation:
    \begin{center}
      Reflection about a line $L$ in $\mathbb{R}^2$
    \end{center}
    Then find an eigenbasis if you can, and thus determine whether the given transformation is diagonalizable. \\ \\
    Below on the left is a plot of a vector $\vec{u}$ that is on $L$ and a vector $\vec{v}$ that is on $L^\perp$, the line perpendicular to $L$. On the right is a plot of these vectors reflected about $L$. \\
    \begin{tikzpicture}
      \begin{axis}[blank_clamped_style]
        \addplot[<->, black, domain=-0.5:1]{0} node[above,pos=1] {$L$};
        \addplot[<->, black, domain=-1:1]({0},{x}) node[right,pos=1] {$L^\perp$};

        \addplot[-Latex, Firebrick2, ultra thick] coordinates {
          (0, 0)
          (0.75, 0)
        } node[above,pos=1] {$\vec{u}$};

        \addplot[-Latex, DodgerBlue2, ultra thick] coordinates {
          (0, 0)
          (0, 0.75)
        } node[right,pos=1] {$\vec{v}$};
      \end{axis}
    \end{tikzpicture}
    \qquad
    \begin{tikzpicture}
      \begin{axis}[blank_clamped_style]
        \addplot[<->, black, domain=-0.5:1]{0} node[above,pos=1] {$L$};
        \addplot[<->, black, domain=-1:1]({0},{x}) node[right,pos=1] {$L^\perp$};

        \addplot[-Latex, Firebrick2, ultra thick] coordinates {
          (0, 0)
          (0.75, 0)
        } node[above,pos=1] {$T(\vec{u})$};
        \addplot[-Latex, DodgerBlue2, ultra thick] coordinates {
          (0, 0)
          (0, -0.75)
        } node[right,pos=1] {$T(\vec{v})$};
      \end{axis}
    \end{tikzpicture} \\
    Notice $T(\vec{u})$ remains the same, so it's eigenvalue is 1. $T(\vec{v})$ remains on $L^\perp$, but is going in the opposite direction, so it's eigenvalue is $-1$. Therefore, the transformation is diagonalizable and
    \begin{align*}
      \text{eigenbasis} = \Set{\vec{u}, \vec{v}} \text{where} \ \vec{u} \in L, \vec{v} \in L^\perp
    \end{align*}
\end{enumerate}

\end{document}
