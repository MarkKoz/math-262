\documentclass[letterpaper,12pt]{article}
\usepackage[utf8]{inputenc}
\usepackage{amsmath, amsfonts}
\usepackage{blkarray, bigstrut}
\usepackage{xcolor}
\usepackage{braket}

% remove spacing around date and author
\usepackage{titling}
\predate{}
\postdate{}
\preauthor{}
\postauthor{}

\author{}
\title{MATH 262 - Homework 7.3}
\date{} % clear date

\definecolor{bookblue}{RGB}{0, 121, 193}

\newcommand{\mat}[1]{
  \begin{bmatrix}
    #1
  \end{bmatrix}
}

\begin{document}

\maketitle

\begin{enumerate}
  \item[2.]
    Let $A = \mat{1 & 1 & 0 \\ 0 & 2 & 2 \\ 0 & 0 & 3}$. \\
    \\
    Find all (real) eigenvalues for $A$. Then find a basis of each eigenspace, and diagonalize $A$, if you can. Do not use technology. \\
    \\
    Since $A$ is a triangular matrix, its eigenvalues are its diagonal entries:
    \begin{align*}
      \lambda_1 &= 1 \ \text{with a multiplicity of} \ 1 \\
      \lambda_2 &= 2 \ \text{with a multiplicity of} \ 1 \\
      \lambda_3 &= 3 \ \text{with a multiplicity of} \ 1
    \end{align*}
    Now, use these eigenvalues to find the basis of each eigenspace. \\
    \begin{align*}
      E_1 &= \text{ker}(A - I_2)
      = \text{ker} \begin{blockarray}{ccc}
        \color{bookblue}{1} & \color{bookblue}{0} & \color{bookblue}{0} \\
        \begin{block}{[ccc]}
          0 & 1 & 0 \bigstrut[t] \\
          0 & 1 & 2 \\
          0 & 0 & 2 \bigstrut[b] \\
        \end{block}
      \end{blockarray}
      = \text{span}\Set{\mat{1 \\ 0 \\ 0}}
    \end{align*}
    \begin{align*}
      E_2 &= \text{ker}(A - 2I_2)
      = \text{ker} \begin{blockarray}{ccc}
        \color{bookblue}{1} & \color{bookblue}{1} & \color{bookblue}{0} \\
        \begin{block}{[ccc]}
          -1 & 1 & 0 \bigstrut[t] \\
          0 & 0 & 2 \\
          0 & 0 & 1 \bigstrut[b] \\
        \end{block}
      \end{blockarray}
      = \text{span}\Set{\mat{1 \\ 1 \\ 0}}
    \end{align*}
    \begin{align*}
      E_3 &= \text{ker}(A - 3I_2)
      = \text{ker} \begin{blockarray}{ccc}
        \color{bookblue}{1} & \color{bookblue}{2} & \color{bookblue}{1} \\
        \begin{block}{[ccc]}
          -2 & 1 & 0 \bigstrut[t] \\
          0 & -1 & 2 \\
          0 & 0 & 0 \bigstrut[b] \\
        \end{block}
      \end{blockarray}
      = \text{span}\Set{\mat{1 \\ 2 \\ 1}}
    \end{align*}
    The vectors $\mat{1 \\ 0 \\ 0}, \mat{1 \\ 1 \\ 0}, \mat{1 \\ 2 \\ 1}$ form an eigenbasis for $A$, so that $A$ is diagonalizable. $A = SBS^{-1}$ with $S = \mat{1 & 1 & 1 \\ 0 & 1 & 2 \\ 0 & 0 & 1}$ and $B = \mat{1 & 0 & 0 \\ 0 & 2 & 0 \\ 0 & 0 & 3}$.

\end{enumerate}

\end{document}
