\documentclass[letterpaper,12pt]{article}
\usepackage[utf8]{inputenc}
\usepackage{amsmath, amsfonts}
\usepackage{amsthm}

% remove spacing around date and author
\usepackage{titling}
\predate{}
\postdate{}
\preauthor{}
\postauthor{}

\author{}
\title{MATH 262 - Homework 4.2a}
\date{} % clear date

\theoremstyle{remark}
\newtheorem*{claim}{Claim}

\begin{document}

\maketitle

\begin{enumerate}
  \item[1.]
    \begin{claim}
      The following function is a linear transformation.
      \begin{align*}
        T(M) = M\begin{bmatrix}
          1 & 3 \\
          3 & 9
        \end{bmatrix}
        \text{from} \ \mathbb{R}^{2 \times 2} \ \text{to} \ \mathbb{R}^{2 \times 2}
      \end{align*}
    \end{claim}
    \begin{proof}
      Take some $M_1,M_2 \in \mathbb{R}^{2 \times 2}$ and $k \in \mathbb{R}$.
      \begin{align*}
        T(M_1) + kT(M_2) &=
        M_1\begin{bmatrix}
          1 & 3 \\
          3 & 9
        \end{bmatrix} +
        kM_2\begin{bmatrix}
          1 & 3 \\
          3 & 9
        \end{bmatrix} \\
        &= (M_1 + kM_2)\begin{bmatrix}
          1 & 3 \\
          3 & 9
        \end{bmatrix} \\
        &= T(M_1 + kM_2)
      \end{align*}
      Notice that the transformation matrix can be trivially factored out of the initial equation. By showing, $T(M_1) + kT(M_2) = T(M_1 + kM_2)$, $T(M)$ has been shown to follow the sum rule and constant-multiple rule. Therefore, it is a linear transformation.
    \end{proof}
\end{enumerate}

\end{document}
