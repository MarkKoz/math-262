\documentclass[letterpaper,12pt]{article}
\usepackage[utf8]{inputenc}
\usepackage{amsmath, amsfonts}
\usepackage{blkarray}
\usepackage{xcolor}
\usepackage{braket}

% remove spacing around date and author
\usepackage{titling}
\predate{}
\postdate{}
\preauthor{}
\postauthor{}

\author{}
\title{MATH 262 - Homework 7.3}
\date{} % clear date

\definecolor{bookblue}{RGB}{0, 121, 193}

\begin{document}

\maketitle

\begin{enumerate}
  \item[1.]
    Let $A = \begin{bmatrix}0 & -1 \\ 1 & 2\end{bmatrix}$. \\
    \\
    Find all (real) eigenvalues for $A$. Then find a basis of each eigenspace, and diagonalize $A$, if you can. Do not use technology. \\
    \\
    First, find the eigenvalues for $A$.
    \begin{align*}
      \text{tr}(A) = 0 + 2 = 2
    \end{align*}
    \begin{align*}
      \text{det}(A) = (0 \cdot 2) - (-1 \cdot 1) = 0 - (-1) = 0 + 1 = 1
    \end{align*}
    \begin{align*}
      0 &= det(A - \lambda I_2) \\
      &= \lambda^2 - \text{tr}(A)\lambda + \text{det}(A) \\
      &= \lambda^2 - 2\lambda + 1 \\
      &= (\lambda - 1)^2
    \end{align*}
    $A$ has one eigenvalue $\lambda_1 = 1$ with a multiplicity of 2. Now, use this eigenvalue to find the eigenspace and its basis. \\
    \begin{align*}
      E_1 &= \text{ker}(A - I_2)
      = \text{ker} \begin{blockarray}{cc}
        \textcolor{bookblue}{-1} & \textcolor{bookblue}{1} \\
        \begin{block}{[cc]}
          -1 & -1 \\
          1 & 1 \\
        \end{block}
      \end{blockarray}
      = \text{span}\Set{
        \begin{bmatrix}
          -1 \\ 1
        \end{bmatrix}
      }
    \end{align*}
    Because only one linearly independent eigenvector can be found, it's not possible to construct an eigenbasis for $A$. Thus, $A$ fails to be diagonalizable.

\end{enumerate}

\end{document}
