\documentclass[letterpaper,12pt]{article}
\usepackage[utf8]{inputenc}
\usepackage{amsmath, amsfonts}
\usepackage{amsthm}

% remove spacing around date and author
\usepackage{titling}
\predate{}
\postdate{}
\preauthor{}
\postauthor{}

\author{}
\title{MATH 262 - Homework 4.2a}
\date{} % clear date

\theoremstyle{remark}
\newtheorem*{claim}{Claim}

\begin{document}

\maketitle

\begin{enumerate}
  \item[7.]
    \begin{claim}
      The following function is a linear transformation.
      \begin{align*}
        T(f(t)) = t(f^\prime(t)) \ \text{from} \ P_2 \ \text{to} \ P_2
      \end{align*}
    \end{claim}
    \begin{proof}
      Take some $f(t),g(t) \in P_2$ and $k \in \mathbb{R}$. Show that $T(f(t)) + kT(g(t)) = T(f(t) + kg(t))$.
      \begin{align*}
        T(f(t)) + kT(g(t)) &= t \cdot f^\prime(t) + t \cdot kg^\prime(t) \\
        &= t(f^\prime(t) + kg^\prime(t)) \\
        &= t(\frac{d}{dt}f(t) + \frac{d}{dt}kg(t)) \\
        &= t \cdot \frac{d}{dt}(f(t) + kg(t)) \\
        &= T(f(t) + kg(t))
      \end{align*}
      $T(f(t))$ has been shown to follow the sum rule and constant-multiple rule. Therefore, it is a linear transformation.
    \end{proof}
\end{enumerate}

\end{document}
