\documentclass[letterpaper,12pt]{article}
\usepackage[utf8]{inputenc}
\usepackage{amsmath, amsfonts}
\usepackage{blkarray, bigstrut}
\usepackage{xcolor}
\usepackage{braket}

% remove spacing around date and author
\usepackage{titling}
\predate{}
\postdate{}
\preauthor{}
\postauthor{}

\author{}
\title{MATH 262 - Homework 7.3}
\date{} % clear date

\definecolor{bookblue}{RGB}{0, 121, 193}

\newcommand{\mat}[1]{
  \begin{bmatrix}
    #1
  \end{bmatrix}
}

\begin{document}

\maketitle

\begin{enumerate}
  \item[3.]
    Let $A = \mat{1 & 1 & 0 \\ 0 & 1 & 1 \\ 0 & 0 & 1}$. \\
    \\
    Find all (real) eigenvalues for $A$. Then find a basis of each eigenspace, and diagonalize $A$, if you can. Do not use technology. \\
    \\
    Since $A$ is a triangular matrix, its eigenvalues are its diagonal entries. It only has one eigenvalue $\lambda_1 = 1$ with a multiplicity of 3. Now, use this eigenvalue to find the basis of the eigenspace. \\
    \begin{align*}
      E_1 &= \text{ker}(A - I_2)
      = \text{ker} \begin{blockarray}{ccc}
        \color{bookblue}{1} & \color{bookblue}{0} & \color{bookblue}{0} \\
        \begin{block}{[ccc]}
          0 & 1 & 0 \bigstrut[t] \\
          0 & 0 & 1 \\
          0 & 0 & 0 \bigstrut[b] \\
        \end{block}
      \end{blockarray}
      = \text{span}\Set{\mat{1 \\ 0 \\ 0}}
    \end{align*}
    Because only one linearly independent eigenvector can be found, it's not possible to construct an eigenbasis for $A$. Thus, $A$ fails to be diagonalizable.

\end{enumerate}

\end{document}
