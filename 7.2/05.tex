\documentclass[letterpaper,12pt]{article}
\usepackage[utf8]{inputenc}
\usepackage{amsmath, amsfonts}

% remove spacing around date and author
\usepackage{titling}
\predate{}
\postdate{}
\preauthor{}
\postauthor{}

\author{}
\title{MATH 262 - Homework 7.2}
\date{} % clear date

\begin{document}

\maketitle

\begin{enumerate}
  \item[5.]
    Let $A = \begin{bmatrix}a & b \\ b & c\end{bmatrix}$ and $a$, $b$, and $c$ are nonzero constants For which values of $a$, $b$, and $c$ does $A$ have two distinct eigenvalues? \\
    \\
    First, calculate the characteristic polynomial and set it equal to 0.
    \begin{align*}
      0 &= \text{det}(A - \lambda I_2) \\
      &= \text{det}\left(
        \begin{bmatrix}
          a - \lambda & b \\
          b & c - \lambda
        \end{bmatrix}
      \right) \\
      &= (a - \lambda)(c - \lambda) - b^2 \\
      &= \lambda^2 - (a + c)\lambda + ac - b^2
    \end{align*}
    Notice that the result is a quadratic equation. Thus, the quadratic formula can be used. For the quadratic formula to yield two real and distinct solutions (i.e. two distinct eigenvalues), the discriminant must be positive.
    \begin{align*}
      (a + c)^2 - 4(ac - b^2) &> 0 \\
      a^2 + 2ac + c^2 - 4ac + 4b^2 &> 0 \\
      a^2 - 2ac + c^2 + 4b^2 &> 0 \\
      (c - a)^2 + 4b^2 &> 0
    \end{align*}
    Notice this inequality holds true for $b \ne 0$, $a,c \in \mathbb{R}$ and for $b = 0$, $c \ne a$. Since all constants are nonzero, the latter can be ignored. Thus, any real values for $a$, $b$, and $c$ will yield two distinct eigenvalues.
\end{enumerate}

\end{document}
