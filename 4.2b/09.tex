\documentclass[letterpaper,12pt]{article}
\usepackage[utf8]{inputenc}
\usepackage{amsmath, amsfonts, amssymb}
\usepackage{amsthm}
\usepackage{braket}

% remove spacing around date and author
\usepackage{titling}
\predate{}{}
\postdate{}
\preauthor{}
\postauthor{}

\author{}
\title{MATH 262 - Homework 4.2b}
\date{} % clear date

\theoremstyle{remark}
\newtheorem*{claim}{Claim}

\newcommand{\detthree}[9]{
  #1\begin{vmatrix}
    #5 & #6 \\
    #8 & #9
  \end{vmatrix}
  - #2\begin{vmatrix}
    #4 & #6 \\
    #7 & #9
  \end{vmatrix}
  + #3\begin{vmatrix}
    #4 & #5 \\
    #7 & #8
  \end{vmatrix} %\\
  % &= #1[#5 \cdot #9 - #6 \cdot #8] \\
  % &\qquad - #2[#4 \cdot #9 - #6 \cdot #7] \\
  % &\qquad + #3[#4 \cdot #8 - #5 \cdot #7]
}

\begin{document}

\maketitle

\begin{enumerate}
  \item[9.]
    \begin{claim}
      If $a$, $b$, and $c$ are distinct real numbers, then the polynomials $(x-b)(x-c)$, $(x-a)(x-c)$, and $(x-a)(x-b)$ must be linearly independent.
    \end{claim}
    \begin{proof}
      First, expand the polynomials.
      \begin{align*}
        (x-b)(x-c) &= bc - (b+c)x + x^2 \\
        (x-a)(x-c) &= ac - (a+c)x + x^2 \\
        (x-a)(x-b) &= ab - (a+b)x + x^2
      \end{align*}
      These polynomials are in $\mathcal{P}_2$, whose standard basis is $\mathfrak{B} = \Set{1, x, x^2}$. Using $\mathfrak{B}$, represent the polynomials as a matrix $A$.
      \begin{align*}
        A &= \begin{bmatrix}
          bc & -(b+c) & 1 \\
          ac & -(a+c) & 1 \\
          ab & -(a+b) & 1
        \end{bmatrix}
      \end{align*}
      To show the polynomials are linearly independent, show det$(A) \ne 0$.
      \begin{align*}
        \text{det}(A) &= \detthree{bc}{-(b+c)}{1}{ac}{-(a+c)}{1}{ab}{-(a+b)}{1} \\
        &= bc[(a + b) - (a + c)] + (b + c)(ac - ab) + ab(a + c) - ac(a + b) \\
        % &= bc(b - c) + (b + c)(ac - ab) + ab(a + c) - ac(a + b) \\
        % &= bc(b - c) + (b + c)(ac - ab) + ba^2 - ca^2 \\
        &= cb^2 - bc^2 + ac^2 - ab^2 + ba^2 - ca^2 \\
        &= (a^2 - ab - ac + bc)(b - c) \\
        &= (a - b)(a - c)(b - c)
      \end{align*}
      Notice that for det$(A) = 0$, at least two values would have to equal 0. However, all three values are distinct, so this is impossible. Therefore, det$(A) \ne 0$ and the polynomials are linearly independent.
    \end{proof}
\end{enumerate}

\end{document}
