\documentclass[letterpaper,12pt]{report}
\usepackage[utf8]{inputenc}
\usepackage{amsmath}

\begin{document}

For $A$ to be invertible, its columns must be linearly independent. To check this, consider the following equation
\begin{align*}
  A\vec{x} &= \vec{0} \\
  \vec{v}_{1}x_{1} + \vec{v}_{2}x_{2} + \ldots + \vec{v}_{n}x_{n} &= \vec{0}
\end{align*}
Let $i = 1$ and $j = 2$ so that the first two columns of $A$ are equal. Then, the equation becomes
\begin{align*}
  \vec{v}_{1}(x_{1} + x_{2}) + \ldots + \vec{v}_{n}x_{n} = \vec{0}
\end{align*}
Notice that to get $\vec{0}$ as a solution, any constant such that $x_{1} = -x_{2}$ will cancel out the first term. Then, the rest of $\vec{x}$ can be zeros. Since there is an infinite number of solutions rather than only the trivial $\vec{x} = \vec{0}$, the columns of $A$ are linearly dependent. Therefore, $A$ is not invertible.
\end{document}
