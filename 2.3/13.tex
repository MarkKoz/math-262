\documentclass[letterpaper,12pt]{report}
\usepackage[utf8]{inputenc}
\usepackage{amsmath}
\usepackage[x11names]{xcolor}

\begin{document}

If $A^{17} = I_2$, then it is false that $A$ must be $I_2$. As a counterexample, let $A$ be a rotation matrix on $\frac{2\pi}{17}$:
\begin{align*}
  A = \begin{bmatrix}
    \cos(\frac{2\pi}{17}) & -\sin(\frac{2\pi}{17}) \\
    \sin(\frac{2\pi}{17}) & \cos(\frac{2\pi}{17})
  \end{bmatrix}
\end{align*}
A single rotation on $2\pi$ would give $I_2$:
\begin{align*}
  \begin{bmatrix}
    \cos(2\pi) & -\sin(2\pi) \\
    \sin(2\pi) & \cos(2\pi)
  \end{bmatrix} = \begin{bmatrix}
    1 & 0 \\
    0 & 1
  \end{bmatrix} = I_2
\end{align*}
If the rotation angle $2\pi$ is divided by 17, then repeating the rotation 17 times would result in a rotation by $2\pi$. This is what is being done when the rotation matrix $A$ is put to the power of 17.

\end{document}
