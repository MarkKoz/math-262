\documentclass[letterpaper,12pt]{article}
\usepackage[utf8]{inputenc}
\usepackage{amsmath, amsfonts}
\usepackage{amsthm}

% remove spacing around date and author
\usepackage{titling}
\predate{}
\postdate{}
\preauthor{}
\postauthor{}

\author{}
\title{MATH 262 - Homework 4.1}
\date{} % clear date

\theoremstyle{remark}
\newtheorem*{claim}{Claim}

\begin{document}

\maketitle

\begin{enumerate}
  \item[14.]
    True or False? Upload your reasoning. If the statement is true give a proof or detailed reason why it is true. If it is false give a counter example. \\
    \\
    \begin{claim}
      If $W_1$ and $W_2$ are subspaces of a linear space $V$, then the intersection $W_1 \cap W_2$ must be a subspace of $V$ as well.
    \end{claim}
    \begin{proof}
      Let $\vec{v}$ and $\vec{w}$ be vectors such that $\vec{v}, \vec{w} \in W_1 \cap W_2$. By the definition of a set intersection, $\vec{v}, \vec{w} \in W_1$ and $\vec{v}, \vec{w} \in W_2$. By the definition of a linear subspace, $\vec{v} + k\vec{w} \in W_1$ and $\vec{v} + k\vec{w} \in W_2$ for some $k \in \mathbb{R}$. Therefore, by the definition of a set intersection, $\vec{v} + k\vec{w} \in W_1 \cap W_2$. \\
      \\
      By the definition of a linear subspace, both $W_1$ and $W_2$ contain the neutral element 0 of $V$. Therefore, by the definition of a set intersection, $W_1 \cap W_2$ also contains the neutral element $0$ of $V$. \\
      \\
      It has been shown that $W_1 \cap W_2$ is closed under linear combinations and contains the neutral element 0 of $V$. Therefore, the definition of a linear subspace is satisified and the claim is true.
    \end{proof}
    \textit{Hint:} Start with $k \in \mathbb{R}$ and $\vec{v}, \vec{w} \in W_1 \cap W_2$. Then, are $\vec{v}$ and $\vec{w}$ both in $W_2$? If so, why is $\vec{v} + k\vec{w} \in W_2$? Can you conclude that $\vec{v} + k\vec{w} \in W_1 \cap W_2$?
\end{enumerate}

\end{document}
